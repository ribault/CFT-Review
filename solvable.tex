

\documentclass[12pt, a4paper]{article}

\raggedbottom

\RequirePackage[l2tabu, orthodox]{nag}

\usepackage[top=20mm,bottom=20mm,left=25mm,right=25mm]{geometry}

%\usepackage{showkeys}

\usepackage{amsmath,amssymb,amsfonts}

\usepackage{centernot}

\usepackage{underbracket}

\usepackage[nottoc]{tocbibind}

\usepackage{pgf}
\usepackage{tikz}

\usepackage{multirow}

\usepackage{pgf}
\usepackage{tikz}

\usepackage{xcolor}

% Syntax: \colorboxed[<color model>]{<color specification>}{<math formula>}
\newcommand*{\colorboxed}{}
\def\colorboxed#1#{%
  \colorboxedAux{#1}%
}
\newcommand*{\colorboxedAux}[3]{%
  % #1: optional argument for color model
  % #2: color specification
  % #3: formula
  \begingroup
    \colorlet{cb@saved}{.}%
    \color#1{#2}%
    \boxed{%
      \color{cb@saved}%
      #3%
    }%
  \endgroup
}

\usepackage[most]{tcolorbox}

\tcbset{
    frame code={}
    center title,
    left=0pt,
    right=0pt,
    top=0pt,
    bottom=0pt,
    colback=green!15,
    colframe=white,
    % width=\dimexpr\textwidth\relax,
    enlarge left by=0mm,
    % boxsep=5pt,
    arc=0pt,outer arc=0pt,
    breakable,
    }

\usepackage{amsthm}

\usepackage{thmtools}
%\usepackage{thm-restate}   This package does not interact very smoothly with \theoremstyle and \listoftheorems

\newtheoremstyle{break}{9pt}{9pt}{\itshape}{}{\bfseries}{}{\newline}{}
\theoremstyle{break}    

\newtheorem{exo}{Exercise}[section]
\newtheorem{hyp}[exo]{Axiom}
\newtheorem{res}[exo]{Result}
\newtheorem{defn}[exo]{Definition}

\makeatletter
\def\ll@exo{%
  \protect\numberline{\csname the\thmt@envname\endcsname}%
  \ifx\@empty\thmt@shortoptarg
    \thmt@thmname
  \else
    \thmt@shortoptarg
  \fi}
\def\l@thmt@exo{} 
\makeatother

\usepackage[colorlinks=true,linktoc=all,linkcolor=black,citecolor=red,urlcolor=blue,backref=page]{hyperref}

\usepackage{etoolbox}
\makeatletter
\patchcmd{\BR@backref}{\newblock}{\newblock(}{}{}
\patchcmd{\BR@backref}{\par}{)\par}{}{}
\makeatother

\title{\bfseries Exactly solvable two-dimensional \\ conformal field theories}

\author{Sylvain Ribault \vspace{2mm}
\\
{\normalsize CEA Saclay, Institut de Physique Th\'eorique}
 \\
 {\footnotesize \ttfamily sylvain.ribault@ipht.fr }
}

\begin{document}


\maketitle


\begin{abstract}
This course will introduce two-dimensional CFT in the bootstrap approach, and sketch the known exactly solvable CFTs with no extended chiral symmetry.
\begin{itemize}
 \item The Virasoro algebra, its representations, Ward identities, fusion rules.
 \item (Generalized) minimal models, Liouville theory, logarithmic minimal models, the $O(n)$ model and more general loop models. Taking limits in the central charge and/or in conformal dimensions. 
 \item Conformal bootstrap methods, analytic and numerical. Generic and degenerate conformal blocks. Crossing symmetry equations and their solutions. 
 \item Exactly known structure constants. Analytic properties of correlation functions. 
\end{itemize}
\end{abstract}

\vspace{5mm}


\textit{
Could make life easier by renouncing logarithms. But they appear naturally when taking limits, and are in principle needed for solving the $O(n)$ model.}

\clearpage

\tableofcontents

\hypersetup{linkcolor=blue}

\numberwithin{equation}{section}
\setcounter{secnumdepth}{2}
\setcounter{tocdepth}{2}
\setcounter{section}{-1}

\section{Introduction}

Sketch map of CFTs according to symmetry: deg. fields, no deg. fields, local conf. sym, global conf. sym.

CFTs that are exactly solvable with known methods = bootstrap, analytic and numerical. Only Virasoro. Do not characterize with twist gap, as symmetry cannot be inferred from spectrum: chiral symmetry fields might be absent from spectrum. Also, affine abelian symmetry can be present in spectrum but absent in interactions. (Is there a strong statement that unitary CFTs do not have these pathologies?)

Forget Coulomb gas, a dead end. For correl fct, applies only if we have two deg. fields? But used to find spectrum of $O(n)$ etc. Larger symmetry allows many generalizations, but tend to be always more complicated. (Liouville not solved from WZW, quite the opposite.) No modular bootstrap. 

Exact solution means analytic expression for three-point structure constants. Here: CFTs that are presumed to be solvable, not necessarily solved. 

Focus on correl fct: no modular bootstrap, no KPZ. These methods are efficient at getting the spectrum but we can get it otherwise. 

This review: only necessary calculations. Derive the recursion for blocks? hard part is coef $R_{r,s}$ but it obeys simple shift equations. Rewrite it in terms of Upsilon functions, and somehow derive it from Liouville theory! But cancellation of double poles hard to argue? And need to determine sign, since we get square of $R_{r,s}$. We can choose any value of the continuous parameters including $c$, but is there a value that makes $R_{r,s}$ simple enough? And what about the $\Delta\to\infty$ prefactors? See Yin et al? See Appendix D of \cite{msz15} for the pillow geometry that underlies the variable $q$.

Use low central charge notations ($\beta$ not $b$) because more examples. Rare are solved CFTs with large $c$.

Relation with \cite{rib16}.

\subsection*{Acknowledgements}


%\end{tcolorbox}

\section{Basic structures}

In this section we will introduce the basic structures of conformal field theory: conformal symmetry, fields, and operator product expansions. In particular we will focus on the two ingredients of exact solvability: \textit{local} conformal symmetry and \textit{degenerate} fields. 

\subsection{The Virasoro algebra and its representations}

\subsubsection{The Virasoro algebra}

Let us consider the Riemann sphere $\overline{\mathbb{C}}=\mathbb{C}\cup \{\infty\}$, equipped with the metric $ds^2 = dzd\bar z$. By definition, a conformal transformation of a Riemannian manifold is a transformation that preserves angles. On any open subset of the Riemann sphere, a map $z\to f(z)$ is conformal if and only if it is holomorphic. It is indeed straightforward to show that any holomorphic map is conformal,
because it transforms the metric into itself, up to a scalar factor: 
\begin{align}
 dzd\bar z\to dfd\bar f = |f'(z)|^2 dzd\bar z\ .
\end{align}
The Lie algebra of infinitesimal conformal transformations on $\mathbb{C}^*= \mathbb{C}-\{0\}$ is called the Witt algebra. It is infinite-dimensional, with the basis 
\begin{align}
 \left(\ell_n\right)_{n\in\mathbb{Z}}  \quad \text{with} \quad \ell_n = -z^{n+1}\frac{\partial}{\partial z}\ , 
\end{align}
and the commutation relations 
\begin{align}
 [\ell_n,\ell_m] = (n-m)\ell_{m+n}\ .
\end{align}
The generators of the Witt algebra include the translation generator $\ell_{-1} = -\frac{\partial}{\partial z}$, and the dilation generator $\ell_0 = -z\frac{\partial}{\partial z}$. In fact, $(\ell_{-1},\ell_0,\ell_1)$ generate the infinitesimal conformal transformations on the full Riemann sphere.
The corresponding Lie group is the group of global conformal transformations $PSL_2(\mathbb{C})$, whose elements act as 
\begin{align}
 z \mapsto \frac{az+b}{cz+d}\quad , \quad (a,b,c,d\in \mathbb{C},\ ad-bc\neq 0)\ .
\end{align}


\subsubsection{Highest-weight representations}

\subsubsection{Null vectors and degenerate representations}

\subsection{Fields and operator product expansions}

Everything chiral here! And outside correlation functions.

\subsubsection{State-field correspondence}

And commutativity of fields.

\subsubsection{Operator product expansion}

OPE Ward identities. Poles at degenerate channel momentums in OPE. 

\subsubsection{Degenerate OPEs and fusion rules}

Degenerate fusion. 

Vanishing NV implies finite fusion is quite easy because descendants (including NV) come with polynomial coefs. Reverse implication is harder, we would need a definition of the fusion product. Or maybe impossible, who know what happens in all possible CFTs? 

\subsection{Conformal symmetry and correlation functions}

\subsubsection{Conformal algebra}

Non-chiral CFT: two times Virasoro.

Representations may not be factorized.

\subsubsection{Interchiral algebra}

\subsubsection{Single-valuedness}


\section{Sketching exactly solvable CFTs}

Diagonal fields. Non-diagonal fields. Get logarithmic fields by degenerate fusion.

Start with GMM. Then focus on rational central charge, get MM. If we take limit, we presumably get log-MM. 

Get D-series, can we get E-series? 

Deduce Liouville with $c\leq 1$ by taking a limit. Then get Liouville for generic $c$. Other limits of diag CFT: RWT. Limit of non-diagonal MM.

Then reduce to one degenerate field. Maximal spectrum is $O(n)$. Depends on $\beta^2$, not $c$. Cannot really argue for Potts, but does not matter much. Anyway, at this stage, we know nothing about global symmetry.

Brownian loop soup: presumably has higher symmetry.

Ashkin--Teller does have higher symmetry, solved wrt Virasoro nevertheless.

Criterion for $b^2\in\mathbb{Q}$ limit to become logarithmic? For 4pt function, criterion about vanishing of residues, which depends on fusion rules being obeyed for NV? In MM case we completely escape log. reps, criterion should account for that.  
For $c>25$, what happens to GMM at rational $b^2$? Concidences of dimensions boil down to $\Delta_{(r,s)}-\Delta_{(r,-s)} \in \mathbb{Z}$ modulo $(p, -q)$. 

\section{Conformal bootstrap and conformal blocks}

Not yet degenerate. 

\section{Degenerate four-point functions}

\subsection{BPZ equations and their solutions}

Non-diagonal, simplify it as much as possible. Non-diagonal solution. 

Loop weights are invariant under shifts, left undetermined. 

$s\to s+1$ shift equations can be violated, cf Potts model. $s\to s+2$ may be derived from $(1,2)$ deg. field but are always obeyed if $(1,3)$ exists. 

Do not try to rederive monodromies of BPZ equations! 

\subsection{Exact solutions for structure constants}

Deduce analytic properties of correlation functions, and limits of CFTs. 

Do we want a synthetic table of known solutions? 

\begin{itemize}
\item Invariance of $C_{(r_1,s_1)(r_2,s_2)(r_3,s_3)}$ under permutations of $1,2,3$
\item Reduction to $C_{P_1,P_2,P_3}$ if $\forall i, r_i=0$ 
 \item $V_{\langle 1,2\rangle}^d$ shift equations $\implies$ $s_i\to s_i+2$  
 \item Parity $\forall i, (r_i,s_i)\to (r_i,-s_i)$, reflection $(r_i,s_i)\to (-r_i,-s_i)$
\end{itemize}


\section{Numerical bootstrap}

Synthesis of what precedes?

Notion of interchiral symmetry. Liouville 4pt is only one interchiral block? Same for MM, including non-diagonal. Distinguish double interchiral from simple interchiral, cf nb of deg. fields. 

Sketch method. Start with Zamolodchikov recursion.


\section{Logarithmic representations and fields} 

Not clear where to put this section. Before conformal blocks or after?

We need log reps. Can deduce them from non-log using degenerate fields. Issue is conceptually less important if we adopt interchiral symmetry, since logarithms only appear for interchiral descendants. 

To get log reps we need associativity of OPE of degenerate field. We might say that this does not belong to the chiral, algebraic part of the review. However, to get conformal blocks we need associativity of OPEs of the energy-momentum tensor. Again, interchiral point of view changes things! And we need bootstrap axioms, including single-valuedness? 

Emergence of logarithms: $\Delta_{P+\frac{\beta}{2}}-\Delta_{P-\frac{\beta}{2}} = \beta P$ integer $\implies P=P_{(0,s)}$. This calculation is a diagonal version of the argument about integer spin that leads to non-diagonal $O(n)$ spectrum.

In $V_{\langle 1,2\rangle}V^N_{(r,0)} = \sum_\pm V^N_{(r,\pm 1)}$ it is not just spins but also dimensions that differ by integers. 

Need we assume that $V^N_{(r,0)}$ is the limit of diagonal fields? Why would we not get $V_{P_{(r,0)}}$? Are there two ways to take the limit? See OPE coefficients. 

\bibliographystyle{cft}
\bibliography{cft}

%\begin{thebibliography}{10}
\expandafter\ifx\csname url\endcsname\relax
  \def\url#1{\texttt{#1}}\fi
\expandafter\ifx\csname urlprefix\endcsname\relax\def\urlprefix{URL }\fi
\providecommand{\eprint}[2][]{\url{#2}}

\bibitem{zz90}
A.~Zamolodchikov, A.~Zamolodchikov (1990 book)\\ {\em {Conformal field theory
  and 2-D critical phenomena}\/}

\bibitem{sch05}
\href{http://arxiv.org/abs/hep-th/0509155}{V.~Schomerus} (2006 review)\\ {\em
  Non-compact string backgrounds and non-rational CFT\/}

\bibitem{fms97}
P.~Di~Francesco, P.~Mathieu, D.~Senechal (1997 book)\\ {\em Conformal field
  theory\/}

\bibitem{nak04}
\href{http://arxiv.org/abs/hep-th/0402009}{Y.~Nakayama} (2004 review)\\ {\em
  Liouville field theory: A decade after the revolution\/}

\bibitem{gab99}
\href{http://arxiv.org/abs/hep-th/9910156}{M.~R. Gaberdiel} (2000 review)\\
  {\em {An Introduction to conformal field theory}\/}

\bibitem{car08}
\href{http://arxiv.org/abs/0807.3472}{J.~Cardy} (2008 review)\\ {\em {Conformal
  Field Theory and Statistical Mechanics}\/}

\bibitem{bs92}
\href{http://arxiv.org/abs/hep-th/9210010}{P.~Bouwknegt, K.~Schoutens} (1993
  review)\\ {\em W symmetry in conformal field theory\/}

\bibitem{tv12}
\href{http://arxiv.org/abs/1202.4698}{J.~Teschner, G.~Vartanov} (2012)\\ {\em
  {6j symbols for the modular double, quantum hyperbolic geometry, and
  supersymmetric gauge theories}\/}

\bibitem{aflt10}
\href{http://arxiv.org/abs/1012.1312}{V.~A. Alba, V.~A. Fateev, A.~V. Litvinov,
  G.~M. Tarnopolsky} (2011)\\ {\em {On combinatorial expansion of the conformal
  blocks arising from AGT conjecture}\/}

\bibitem{flno09}
\href{http://arxiv.org/abs/0902.1331}{V.~A. Fateev, A.~V. Litvinov, A.~Neveu,
  E.~Onofri} (2009)\\ {\em {Differential equation for four-point correlation
  function in Liouville field theory and elliptic four-point conformal
  blocks}\/}

\bibitem{sch03}
\href{http://arxiv.org/abs/hep-th/0306026}{V.~Schomerus} (2003)\\ {\em Rolling
  tachyons from Liouville theory\/}

\bibitem{zam05}
\href{http://arxiv.org/abs/hep-th/0505063}{A.~B. Zamolodchikov} (2005)\\ {\em
  On the three-point function in minimal Liouville gravity\/}

\bibitem{rw01}
\href{http://arxiv.org/abs/hep-th/0107118}{I.~Runkel, G.~M.~T. Watts} (2001)\\
  {\em A non-rational CFT with c = 1 as a limit of minimal models\/}

\bibitem{zz95}
\href{http://arxiv.org/abs/hep-th/9506136}{A.~B. Zamolodchikov, A.~B.
  Zamolodchikov} (1996)\\ {\em Structure constants and conformal bootstrap in
  Liouville field theory\/}

\bibitem{tes03b}
\href{http://arxiv.org/abs/hep-th/0303150}{J.~Teschner} (2004)\\ {\em A lecture
  on the Liouville vertex operators\/}

\bibitem{cer12}
\href{http://arxiv.org/abs/1209.3984}{L.~Chekhov, B.~Eynard, S.~Ribault}
  (2013)\\ {\em {Seiberg-Witten equations and non-commutative spectral curves
  in Liouville theory}\/}

\bibitem{rib09}
\href{http://arxiv.org/abs/0912.4481}{S.~Ribault} (2010)\\ {\em {Minisuperspace
  limit of the AdS3 WZNW model}\/}

\bibitem{rib08b}
\href{http://arxiv.org/abs/0811.4587}{S.~Ribault} (2009)\\ {\em {On sl3
  Knizhnik-Zamolodchikov equations and W3 null-vector equations}\/}

\bibitem{rib08}
\href{http://arxiv.org/abs/0803.2099}{S.~Ribault} (2008)\\ {\em {A family of
  solvable non-rational conformal field theories}\/}

\bibitem{tes97a}
\href{http://arxiv.org/abs/hep-th/9712256}{J.~Teschner} (1999)\\ {\em On
  structure constants and fusion rules in the $SL(2,\mathbb{C})/SU(2)$ {WZNW}
  model\/}

\bibitem{rt05}
\href{http://arxiv.org/abs/hep-th/0502048}{S.~Ribault, J.~Teschner} (2005)\\
  {\em $H_3^+$ correlators from Liouville theory\/}

\bibitem{tes97b}
\href{http://arxiv.org/abs/hep-th/9712258}{J.~Teschner} (1999)\\ {\em The
  mini-superspace limit of the {SL(2,C)/SU(2) WZNW} model\/}

\bibitem{mo00a}
\href{http://arxiv.org/abs/hep-th/0001053}{J.~M. Maldacena, H.~Ooguri} (2001)\\
  {\em Strings in {$AdS_3$ and $SL(2,\mathbb{R})$ WZW model. I}\/}

\bibitem{gab01b}
\href{http://arxiv.org/abs/hep-th/0105046}{M.~R. Gaberdiel} (2001)\\ {\em
  {Fusion rules and logarithmic representations of a WZW model at fractional
  level}\/}

\bibitem{rib05}
\href{http://arxiv.org/abs/hep-th/0507114}{S.~Ribault} (2005)\\ {\em
  Knizhnik-Zamolodchikov equations and spectral flow in $AdS_3$ string
  theory\/}

\end{thebibliography}


\end{document}


\appendix




\end{document}

